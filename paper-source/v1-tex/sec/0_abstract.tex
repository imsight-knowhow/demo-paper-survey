\begin{abstract}
\nonumfootnote{\noindent $^*$ denotes equal contribution}
\nonumfootnote{This work was done when Toshiki was a remote intern at UC Berkeley}
We introduce StreamDiffusion, a real-time diffusion pipeline designed for streaming image generation. Existing diffusion models are adept at creating images from text or image prompts, yet they often fall short in real-time interaction. This limitation becomes particularly evident in scenarios involving continuous input, such as augmented/virtual reality, video game graphics rendering, live video streaming, and broadcasting, where high throughput is imperative. 
StreamDiffusion tackles this challenge through a novel pipeline-level system design. It employs unique strategies like batching the denoising process (Stream Batch), residual classifier-free guidance (R-CFG), and stochastic similarity filtering (SSF). Additionally, it seamlessly integrates advanced acceleration technologies for maximum efficiency.
Specifically, Stream Batch reformulates the denoising process by eliminating the traditional wait-and-execute approach and utilizing a batching denoising approach, facilitating fluid and high-throughput streams. This results in 1.5x higher throughput compared to the conventional sequential denoising approach. R-CFG significantly addresses inefficiencies caused by repetitive computations during denoising. It optimizes the process to require minimal or no additional computations, leading to speed improvements of up to 2.05x compared to previous classifier-free methods. Besides, our stochastic similarity filtering dramatically lowers GPU activation frequency by halting computations for static image flows, achieving a remarkable reduction in computational consumption—2.39 times on an RTX 3060 GPU and 1.99 times on an RTX 4090 GPU, respectively. The synergy of our proposed strategies with established acceleration technologies enables image generation to reach speeds of up to 91.07 fps on a single RTX 4090 GPU, outperforming the throughput of AutoPipeline, developed by Diffusers, by more than 59.56x. 
\end{abstract}
